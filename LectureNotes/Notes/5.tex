\section{Dynamic Programming}
From smaller subproblems to larger subproblems. 
\begin{enumerate}[-]
    \item \textbf{DAGs:} guarantees topological order, i.e. each  sub-problem only depends on previous sub-problem
    \item \textbf{Induction:} solve sub-problems in topological order, implemented by a for loop.
    \item \textbf{Memoization:} (not memorization)"backward" recursion with "storing"
\end{enumerate}
% \begin{figure}
%     \centering
%     \includegraphics[width=0.5\linewidth]{DP.png}
% \end{figure}

Different from Divide and Conquer, where the scale decreases exponentially by a \textbf{'magic' merge operation}, 
DP reduces the scale a little bit.

Difference between Induction and Memoization: Theoretically, time-complexity is the same. 
Memoization has the advantage of not needing to compute \textbf{topological order}, and some \textbf{not-computed sub-problems}.
But in coding, 'throwing recursive call' is time costing.


\subsection{Longest Increasing Subsequence}
Formation of a DAG: establish a node i($i\in [1,n]$) for each element $a_i$, if $i<j$ and $a_i<a_j$, then $(a_1,a_j)\in E$.
Then the problem is to find the longest path in the DAG.
Construct L(i) by:
\[
    L(j)=1+\max{L(i):(i,j)\in E}\]
\textbf{Time Complexity} is $O(|V||E|)\leq O(n^2)$.\\
\textbf{Correctness of Algorithm:} 

We'll prove that L(i) is the length of LIS ending at $a_i$.

For base step, L(0)=0 is the length of LIS for an empty sequence.

For induction step, assume that L(i) is the length of LIS ending at $a_i$, 
Let S be the LIS ending at $a_i$ and $T=S\backslash \{a_i\}$, the last number of $T$ is $a_t$.
Prove by $LIS[i]\geq |S|$, and $LIS[i]\leq |S|$.
Since $T$ is a LIS ending at $a_t$, $|T|=L(t)$, so $|S|=L(t)+1$. $L(i)=1+\max_{a_j<a_i,j<i}\{L(j)\}\geq L(t)+1=|S|$.

On the other hand, $L(i)=1+\max_{a_j<a_i,j<i}{L(j)+1}\leq 1+|T|=|S|$, so $L[i]\leq |S|$.

Therefore, $L(i)=|S|$.

\subsection{Edit Distance}
Edit distance is the minimum number of edits—insertions, deletions, and substitutions of characters—needed to transform
the first string into the second.
We denote $E(i,j)$ as the editing distance between $x[1..i]$ and $y[1..j]$.
\[
    E(i,j)=\min(1+E(i-1,j)+E(i,j-1)+E(i-1,j-1))
\]
diff(i,j)=0 iff x[i]=y[j].
Forming a $m \times n$ table, and each cell is computed only once, which results in \textbf{Time Complexity} of $O(mn)$.

\begin{algorithm}
    \caption{Edit Distance}
    \KwIn{Two strings $x[1..m]$ and $y[1..n]$}
    \KwOut{Edit distance $E(m,n)$}
    \For{$i=1$ to $m$}{
        $E(i,0)=i$\;
    }
    \For{$j=1$ to $n$}{
        $E(0,j)=j$\;
    }
    \For{$i=1$ to $m$}{
        \For{$j=1$ to $n$}{
            $E(i,j)=\min(1+E(i-1,j),1+E(i,j-1),diff(i,j)+E(i-1,j-1))$\;
        }
    }
    \Return{$E(m,n)$}\;
\end{algorithm}
Palindrome is a string that reads the same backward as forward, e.g., madam, racecar, abba, etc.
Likewize, in the palindrome problem,
$H(i,j)$ to denote the biggest palindrome length.
\begin{equation}
H(i,j)= 
\begin{cases}
H(i+1,j-1)+2, & \text{if } A[i]=A[j],\\
\max\{H(i,j-1),H(i+1,j)\}, & \text{o.w.}
\end{cases}
\end{equation}

\subsection{Nnapsack Problem}
Denote $s[i]=v[i]/c[i]$, $s$ means the ratio of value to cost.
\begin{enumerate}
    \item indivisibility\\
    If it is divisible, if you choose the object with the largest xingjiabi each time, it is optimal, else you can switch two objects and get a better one, which results in a better solution 

    If it is indivisible, it is essentially an NP-Hard problem. As a problem where r of all objects are equal and v=totalSum/2 is a Partition Problem.
\end{enumerate}

For \textbf{time-complexity}, it is not a polynomial problem, with $O(nW)$, where W is a numerical value rather than size.

We can control the precision of $W$ by rounding, but the new optim solution you obtain may be infeasible in the original problem.

Howvever, if you round value, the feasibility of solutions doesn't change. 
Let $V=\max v_i$, $OPT\leq nV$. Denote $A[i,v]$ as the minimum cost cost(S) if we select till the $i^{th}$ item with exactly value $v$.
\begin{equation}
    A(i+1,v)= 
    \begin{cases}
    \min{A[i,v],c_{i+1}+A[i,v-v_{i+1}]} if v[i+1]<v\\
    A(i,v) & \text{o.w.}
    \end{cases}
\end{equation}

\[
    f[i][v]=\max{f[i-1][v],f[i-1][v-c[i]]+w[i]}\]
Where $f[i][v]$ means putting the first $i$ objects into a $v$-capacity Knapsack. For every object $i$, you either put it in the bag, where the left capacity is $v-c[i]$, or not put it in, which is equal to $f[i-1][v]$.

\begin{remark}
    Fully Polynomial Time Approximation Scheme(FPTAS) gives a $(1-\epsilon)-$approximation with running time polynomial in term of $n$ and $\frac{1}{\epsilon}$

    Algorithms that can \textbf{scale down} the problem to constant length and use naive explore algorithm are commonly FPTAS.

    Polynomial Time Approximation Scheme(PTAS)
    gives a $(1-\epsilon)-$approximation with running time polynomial in term of $n$ and constant $\epsilon$.

    Some have fixed constant ratio
\end{remark}

\subsection{Floyd-Warshall Algorithm and All pairs shortest paths}
Floyd-Warshall Algorithm finds the shortest path between any two vertice on a directed, real number weighted(but no negetive-weighted cycles) graph.

Run bellman-Ford for all vertices takes $O(|V|^2|E|)$, but the algorithm reduces to $O(|V|^3)$.\\
Denote $dist_k(i,j)$ as the distance between $i$ and $j$ by using intermediate vertices from ${1,\ldots,k}$, not including start and end vertice. In implementation, also denoted as $d(i,j,k)$.

\textbf{Algorithm:} Gradually expand the set of permissible intermediate nodes by updateing one node at a time, updating the shortest path lengths at each stage. Eventually this set grows to all of V, at which point all vertices are allowed to be on all paths, and we have found the true shortest paths between vertices of the graph.

\[
    dist_{k_0+1}(i,j)=\min{dist_{k_0+1}(i,j),dist_{k_0}(i,k_0+1)+dist_{k_0}(k_0+1,j)}\]
$d(i,k,k-1)=d(i,k,k)$ holds because no negative cycle exists. 

\begin{algorithm}
    \caption{Floyd-Warshall Algorithm}
    For each pair $i,j \in V$, set 
    \begin{equation}
        d(i,j,0)=
        \begin{cases}
            0 & \text{if } i=j\\
            w(i,j) & \text{if } (i,j)\in E\\
            \infty & \text{o.w.}
        \end{cases}
    \end{equation}
    \For{$k=1$ to $n$}{
        \For{$i=1$ to $n$}{
            \For{$j=1$ to $n$}{
                $d(i,j,k)=\min(d(i,j,k-1),d(i,k,k-1)+d(k,j,k-1))$\;
                \# or similarly $d(i,j)=\min(d(i,j),d(i,k)+d(k,j))$
            }
        }
    }
\end{algorithm}
A trick is that we can reduce \textbf{space complexity} to $O(|V|^2)$ by using two $|V|\times |V|$ matrices, one for $k$ and one for $k-1$.
original: $d(i,j,k)=\min(d(i,j,k-1),d(i,k,k-1)+d(k,j,k-1))$
Now: $d(i,j)=\min(d(i,j),d(i,k)+d(k,j))$
Whether $d(i,k)$ and $d(k,j)$ are updated doens't matter, as $d(i,k,k-1)=d(i,k,k)$

\subsection{DP with Exhaustice Searches}
Given an $m\times n$ chessboard, how many different
ways we can place “kings” such that no pair of them are in
the attack range of each other.

Naive exhaustive search takes $O(2^{mn})$, and a DP-based algorithm takes $O(4^nm)$



\subsection{maximum independent set on trees}
A set of vertices $S$ is an independent set if no two vertices in $S$ are adjacent.
Consider a tree structure

\begin{figure}
    \centering
    \includegraphics[width=0.5\linewidth]{Notes/fig/maxIndependentSet.png}
\end{figure}
\textbf{Time complexity} Each $H(u)$ is looked up exactly twice, on for its parent and one for its grandparent. This is faster than naive case, which is $O(2^n)$, as it contains many sub-optim solutions.
Greedy Algorithm? Choose the leafs and delete them and their parents.
If the leaf node ins't chosen in the optimal solution, then its parent must be chosen, or either choosing it or its parent adds the maximum independent set.

